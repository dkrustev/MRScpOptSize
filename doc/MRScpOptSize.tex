\documentclass[submission,copyright,creativecommons]{eptcs}
\providecommand{\event}{VPT 2020} % Name of the event you are submitting to
\usepackage{breakurl}             % Not needed if you use pdflatex only.
\usepackage{underscore}           % Only needed if you use pdflatex.

\usepackage{listings}
\lstset{
  language=Caml,
  defaultdialect=[Objective]Caml,
  basicstyle=\scriptsize,
  frame=tb
  %	basicstyle=\small\ttfamily,
  %	commentstyle=\textit,
  %	keywordstyle=\bfseries
}

\usepackage{stmaryrd} % for \{ll,rr}bracket	

\title{Optimizing Program Size Using Multi-result Supercompilation}
\author{Dimitur Nikolaev Krustev
\institute{IGE+XAO Balkan\\ Sofia, Bulgaria}
\email{\quad dkrustev@ige-xao.com}
}
\def\titlerunning{Optimizing Program Size Using MRSC}
\def\authorrunning{D.N. Krustev}
\begin{document}
\maketitle

\begin{abstract}
This is a sentence in the abstract.
This is another sentence in the abstract.
This is yet another sentence in the abstract.
This is the final sentence in the abstract.
\end{abstract}

\section{Introduction}

Supercompilation was invented by Turchin \cite{TurchinSupercompilerConcept} and has found numerous
applications, such as program optimization\cite{XX,YY}, program verification \cite{Klyuchnikov},
$\ldots$.

Supercompilation performs very powerful program transformation by simulating
the actual execution of the input program on a whole set of possible inputs
simultaneously.
The flip side of this power is that the behavior of supercompilation -- 
with respect to both transformation time and result size --
can be very unpredictable.
This fact makes supercompilation problematic for including as an
optimization step of a standard compiler, for example.
Measures have been proposed to make supercompilation more
well-behaved, both in execution time and result size \cite{bolingbroke2011improving,Jonsson2011Taming}.
These proposals are all based on a combination of specially crafted and
empirically fine-tuned heuristics.
The main goal of the present study is to experiment with a more
principled approach to optimizing the size of the program resulting
from supercompilation.
This approach is based on a couple of key ideas:
\begin{itemize}
  \item use multi-result supercompilation\footnote{often abbreviated as \emph{MRSC} from now on} 
    to systematically explore a large set of different generalizations during 
    the transformation process, 
    leading to different trade-offs between performed optimizations and code explosion;
  \item carefully select a generalization scheme, which can avoid all forms of
    code duplication if applied systematically;
  \item re-use ideas from Grechanik et al. \cite{Romanenko2014StagedMRSC} to compactly represent and efficiently
    explore the set of programs resulting from multi-result supercompilation.
\end{itemize}
We outline the main ideas of multi-result supercompilation --
as well as the specific approach to its implementation that we use --
in Sec. \ref{sec:MRSCSummary}. We then describe the main contributions of this study:
\begin{itemize}
  \item We define a particular strategy for generalization during MRSC (Sec. \ref*{sec:Generalize}), which:
    \begin{itemize}
      \item avoids any risk of code duplication (when applied);
      \item avoids unnecessary increase of the search space of possible
        transformed programs, which MRSC must explore
    \end{itemize}
  \item We analyze the performance of the proposed strategy on several
    simple examples (Sec. \ref*{sec:EmpEval}).
\end{itemize}

\section{Summary of Multi-result Supercompilation}\label{sec:MRSCSummary}

Most supercompilers perform generalization of the current configuration only
when forced to do so by the whistle.
The only subtlety, as suggested by S{\o}rensen et al. \cite{PosScp} is to select
whether to generalize the lower or the upper configuration of the two that
have caused the whistle.
One of the key insights behind multi-result supercompilation is that the place
where the whistle has blown is not always the best place to make a generalization.
The proposed solution is radical: do not generalize when the whistle has blown,
but at any driving step, where a suitable generalization exists;
pursue in parallel all possible combinations of driving/generalizing 
the current configuration and its descendants.
This parallel analysis of alternative configurations is the source
of the multiple resulting programs.

Our implementation of multi-result supercompilation mostly follows the same
generic framework used in \cite{Romanenko2014StagedMRSC,Krustev2014MRSC}.
It offers some important simplifications compared by the original work
of Klyuchnikov et al. \cite{KlyuchnikovMRSCBranch,MRSCToolkit}:
\begin{itemize}
  \item It is based on big-step driving.
  \item All the set of transformed programs is represented compactly
    in a tree-like data structure, which further permits not only
    recovering the full set of configuration graphs, but also
    performing efficiently some kinds of queries on this set.
\end{itemize}
The compact representation of the set of graphs is shown in Fig. \ref{fig:GraphSet}.
We use direct excerpts of the F\# code of the implementation, but hopefully
they will be readable by anyone familiar with other functional languages such as OCaml and Haskell.
Another important caveat is that we have implemented MRSC for programs in a specific language,
so some details will only become clear once we introduce this language.
Still, the details of the language are not important for understanding the core
MRSC algorithm; indeed, such details are successfully abstracted away in \cite{Romanenko2014StagedMRSC,Krustev2014MRSC}.
We prefer to show excerpts from our actual implementation for concreteness.
This representation is termed \emph{lazy graph} by Grechanik et al. \cite{Romanenko2014StagedMRSC},
and can be viewed as a domain-specific language (DSL) describing the construction of the
complete set of configuration graphs produced by multi-result supercompilation.
\begin{itemize}
  \item Node \verb|GSNone| is used when the whistle has blown. 
    It represents an empty set of configuration graphs;
  \item Node \verb|GSFold| is used when folding is possible. 
    It gives the relative distance to the upper node to which we fold, 
    plus the renaming, which makes the folded configurations compatible.
  \item Node \verb|GSBuild| is the most complicated one, representing a list
    of alternative developments (driving or generalization) of the current configuration.
    Each alternative, in turn, gives rise to a list of new configurations to explore,
    and hence, to a list of nested graph sets.
\end{itemize}

\begin{figure}
\begin{lstlisting}
type GraphSet =
  | GSNone
  | GSFold of MConf * int * list<VarName * VarName>
  | GSBuild of MConf * list<list<GraphSet>>

let rec gset2graphs (gs: GraphSet) : seq<MConf * ConfGraph> =
  match gs with
  | GSNone -> Seq.empty
  | GSFold(conf, n, ren) -> seq{ yield (conf, CGFold(n, ren)) }
  | GSBuild(conf, xss) ->
    xss
    |> Seq.collect (fun xs -> xs |> Seq.map gset2graphs |> Seq.cartesian)
    |> Seq.map (fun cgs -> (conf, buildGraph (snd conf) cgs))
\end{lstlisting}
\caption{Representation and Expansion of Graph Sets}
\label{fig:GraphSet}
\end{figure}

The semantics of this DSL is shown in the same Fig. \ref{fig:GraphSet} as
a function \verb|gset2graphs| expanding a \verb|GraphSet| into a sequence 
of configuration graphs.
Note the use of \verb|Seq.cartesian| to compose the sub-graphs of 
the graph node of each alternative configuration.

\begin{figure}
\begin{lstlisting}
type private HistEntryKind = HEGlobal | HELocal

let rec private mrScpRec (defs: Defs) (nestLvl: int) (hist: 
  list<HistEntryKind * int * Exp>) (conf: MConf) : GraphSet =
  let (_, e) = conf
  let relevantHist hek hist =
    match hek with
    | HELocal -> hist |> List.takeWhile (fun (hek, _, _) -> hek = HELocal)
    | HEGlobal -> hist |> List.filter (fun (hek, _, _) -> hek = HEGlobal)
  match List.tryFindSome (fun (_, _, e1) -> renaming e1 e) hist with
    | Some ((_, lvl, _), ren) -> GSFold(conf, nestLvl - lvl, ren)
    | None ->
      let rs = multiDriveSteps defs e
      let hek = if List.tryFind (fun mdsr -> 
                  match mdsr with MDSRCases _ -> true | _ -> false) rs = None
                then HELocal else HEGlobal
      let relHist = relevantHist hek hist
      match List.tryFind (fun (_, _, e1) -> homeomorphicEmbedding e1 e) relHist with
      | Some _ -> GSNone
      | None ->
        let confss = rs |> List.map (fun mdsr -> 
          mdsrSubExps mdsr |> List.map (fun e -> (mdsr, e)))
        let newHist = (hek, nestLvl, e)::hist
        GSBuild(conf, List.map (List.map (mrScpRec defs (nestLvl + 1) newHist)) confss)

let mrScp (defs: Defs) (e: Exp) : GraphSet =
  mrScpRec defs 0 [] (MDSRUnfold e, e)
\end{lstlisting}
\caption{Main MRSC Algorithm}
\label{fig:MRSCAlg}
\end{figure}

The main MRSC algorithm -- the one that builds the graph set of a given initial
configuration -- is shown in Fig. \ref{fig:MRSCAlg}.
We can ignore the details about splitting the configuration history into local
and global one -- they mostly follow established heuristics as in S{\o}rensen et al. \cite{PosScp}.
The overall approach is simple - if folding is possible, we produce a fold node 
and stop pursuing the current configuration.
Otherwise we check the whistle -- in our case, the now standard homeomorphic embedding relation.
If it blows, we stop immediately with an empty set of resulting graphs.
When there is neither folding nor a whistle, we continue analyzing the execution of the
current configuration -- based on 2 language-specific functions:
\begin{itemize}
  \item \verb|multiDriveSteps| returns a number of alternatives for the current configuration --
    either a driving step, or (possibly several different forms of) generalization.
  \item \verb|mdsrSubExps| returns -- for a given alternative produced by the previous function --
    the list of sub-configurations (in our case -- sub-expressions) that must be subjected
    to further analyzis.
\end{itemize}
The implementation of both functions is described in Sec. \ref{sec:Generalize}.
Once we have this list of lists of sub-configurations, we simply apply the same
algorithm recursively, but with extended history.
Readers familiar with the implementation details of other supercompilers are
invited to compare them to the simplicity of this MRSC approach.

\section{Generalization Approach}\label{sec:Generalize}

\subsection{Programming Language}

The object language we consider is a first-order functional language with ordinary and 
pattern-matching function definitions.
Its syntax is summarized in Fig. \ref{fig:SLLsyntax}.
A very similar language is often used in many introductions to positive
supecompilation \cite{PosScp,KlyuchnikovTMR}.
A notable restriction in our case is that we omit if-expressions and a built-in generic equality.
We assume the language has a call-by-name semantics, as this case is easiest to supercompile,
but we skip a more detailed presentation of the semantics of the language, assuming it mostly obvious.

\begin{figure}
\begin{tabular}[t]{l r l@{\hspace{20pt}} l}
$e$ & ::= && expression   \\
    && $x$ & variable     \\
    && $a(e_1,\ldots,e_n)$ & call \\
$a$ & ::= && call kind    \\
    && $C$ & constructor  \\
    && $f$ & function     \\
$p$ & ::= & $C(x_1, $\ldots$, x_n)$ & pattern \\
$d$ & ::= && function definition \\
    &         & $f(x_1, \ldots, x_n) = e$ & ordinary function \\
    & $\mid$  & $g(p_1, y_1, \ldots, y_m) = e_1$ & pattern-matching function \\
    &         & $\ldots$ & \\
    &         & $g(p_n, y_1, \ldots, y_m) = e_n$ & \\
$P$ & ::= & $d_1, \ldots, d_n$ & program
\end{tabular}
\caption{Object Language Syntax}
\label{fig:SLLsyntax}
\end{figure}

\subsection{Driving}

Let us recall how driving looks like for this simple language, in the case of standard (positive) supercompilation.
As a technical device, we define a single step of driving, producing a
result of type \verb|DriveStepResult|, as defined in Fig. \ref{fig:DriveStepResult}

\begin{figure}
\begin{lstlisting}
type DriveStepResult =
  | DSRNone
  | DSRCon of ConName * list<Exp>
  | DSRUnfold of Exp
  | DSRCases of VarName * list<Pattern * Exp>
\end{lstlisting}
\caption{Result of a Single Step of Driving}
\label{fig:DriveStepResult}
\end{figure}

\begin{itemize}
  \item We cannot drive a variable any further: $\mathtt{drive} \llbracket x \rrbracket = \mathtt{DSNone}$;
  \item Driving a constructor results in a constructor node with all arguments available for
    further driving: $\mathtt{drive} \llbracket C(e_1, \ldots, e_n) \rrbracket = \mathtt{DSCon}(C, e_1, \ldots, e_n)$;
  \item If we stumble upon a call to an ordinary function, we simply unfold its definition:
    $\mathtt{drive} \llbracket f(e_1, \ldots, e_n) \rrbracket = \mathtt{DSUnfold}(e [ x_1\rightarrow e_1, \ldots, x_n\rightarrow e_n ])$,
    where $f(x_1, \ldots, x_n) = e \in P$ and $ [ x_1\rightarrow e_1, \ldots, x_n\rightarrow e_n ] $
    denotes simultaneous substitution;
  \item The most intersting cases concern a call to a pattern-matching function, as the situation is
    differnt depending on the kind of the first argument:
    \begin{itemize}
      \item $\mathtt{drive} \llbracket g(C(e'_1, \ldots, e'_m), e_1, \ldots, e_n) \rrbracket =$
        $\mathtt{DSUnfold}(e [ x_1\rightarrow e'_1, \ldots, x_m\rightarrow e'_m, y_1 \rightarrow e_1, \ldots, y_n \rightarrow e_n ])$
        where $g(C(x_1, \ldots, x_m), y_1, \ldots, y_n) \in P$;
      \item $\mathtt{drive} \llbracket g(x, e_1, \ldots, e_n) \rrbracket =$
        $\mathtt{DSCases}(x, \mathtt{propagate}(x, p_1, (e_1, \ldots, e_n), e'_1),$ $\ldots,$ \\
          $\mathtt{propagate}(x, p_m, (e_1, \ldots, e_n), e'_m))$
        where $g(p_1, y_1, \ldots, y_n) = e'_1, \ldots, g(p_m, y_1, \ldots, y_n) = e'_m \in P$
        and $\mathtt{propagate}$ performs positive information propagation by substituting (a suitable renaming of) $p_i$ for $x$ 
        in the corresponding branch $i$;
      \item $\mathtt{drive} \llbracket g(f(e'_1, \ldots, e'_m), e_1, \ldots, e_n) \rrbracket =$
        $\mathtt{dsrMap}(\llbracket g(\bullet, e_1, \ldots, e_n) \rrbracket, \mathtt{drive} \llbracket f(e'_1, \ldots, e'_m) \rrbracket)$
        where $\mathtt{dsrMap}$ transforms a driving step result by splicing it in an expression with a hole.
    \end{itemize}
\end{itemize}
We deliberately omit many low-level details in this description, as they are well-known and can be found
in most introductions to positive supercompilation \cite{PosScp,KlyuchnikovTMR}.
Using this definition of driving, plus the usual definitions of folding, whistle, and generalization, 
we can build configuration graphs of the form shown in Fig. \ref{fig:ConfGraph}.

\begin{figure}
\begin{lstlisting}
type ConfGraph =
  | CGLeaf of Exp
  | CGCon of ConName * list<ConfGraph>
  | CGUnfold of ConfGraph
  | CGCases of VarName * list<Pattern * ConfGraph>
  | CGFold of int * list<VarName * VarName>
  | CGLet of list<VarName * ConfGraph> * ConfGraph
\end{lstlisting}
\caption{Representation of a Configuration Graph}
\label{fig:ConfGraph}
\end{figure}



\section{Empirical Evaluation}\label{sec:EmpEval}

\section{Future Work}

\section{Related Work}

\section{Conclusions}

%\nocite{*}
\bibliographystyle{eptcs}
\bibliography{MRScpOptSize}

\end{document}
