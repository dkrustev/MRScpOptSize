\documentclass[submission,copyright,creativecommons]{eptcs}
\providecommand{\event}{VPT 2020} % Name of the event you are submitting to
\usepackage{breakurl}             % Not needed if you use pdflatex only.
\usepackage{underscore}           % Only needed if you use pdflatex.

\title{Optimizing Program Size Using Multi-result Supercompilation}
\author{Dimitur Nikolaev Krustev
\institute{IGE+XAO Balkan\\ Sofia, Bulgaria}
\email{\quad dkrustev@ige-xao.com}
}
\def\titlerunning{Optimizing Program Size Using MRSC}
\def\authorrunning{D.N. Krustev}
\begin{document}
\maketitle

\begin{abstract}
This is a sentence in the abstract.
This is another sentence in the abstract.
This is yet another sentence in the abstract.
This is the final sentence in the abstract.
\end{abstract}

\section{Introduction}

Supercompilation was invented by Turchin\cite{TurchinSupercompilerConcept} and has found numerous
applications, such as program optimization\cite{XX,YY}, program verification\cite{Klyuchnikov},
$\ldots$.

Supercompilation performs very powerful program transformation by simulating
the actual execution of the input program on a whole set of possible inputs.
The flip side of this power is that the behavior of supercompilation -- 
with respect to both transformation time and result size --
can be very unpredictable.
This fact makes supercompilation problematic for including as an
optimization step of a standard compiler, for example.
Measures have been proposed to make supercompilation more
well-behaved, both in execution time and result size\cite{SPJBolingbroke,JohnsonNordlander}.
These proposals are all based on a combination of specially crafted and
empirically fine-tuned heuristics.
The main goal of the present study is to experiment with a more
principled approach to optimizing the size of the program resulting
from supercompilation.
This approach is based on a couple of key ideas:
\begin{itemize}
  \item use multi-result supercompilation\footnote{often abbreviated as \emph{MRSC} from now on} 
    to systematically explore a large set of different generalizations during 
    the transformation process, 
    leading to different trade-offs between performed optimizations and code explosion;
  \item carefully select a generalization scheme, which can avoid all forms of
    code duplication if applied systematically;
  \item re-use ideas from Grechanik\cite{BigStepMRSC} to compactly represent and efficiently
    explore the set of programs resulting from multi-result supercompilation.
\end{itemize}

\section{Summary of Multi-result Supercompilation}

\section{Generalization Approach}

\section{Related Work}

%\nocite{*}
\bibliographystyle{eptcs}
\bibliography{MRScpOptSize}

\end{document}
